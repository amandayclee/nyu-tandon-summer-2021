\documentclass[11pt]{article}
\usepackage{fullpage}
\usepackage{amsmath,amsfonts,amsthm,amssymb}
\usepackage{url}
\usepackage[demo]{graphicx}
\usepackage{caption} 
\usepackage{algpseudocode}
\usepackage{bbm}
\usepackage{float}
\usepackage{framed}
\usepackage{enumerate}
\usepackage{color}
\usepackage[colorlinks=true, linkcolor=red, urlcolor=blue, citecolor=blue]{hyperref}
\usepackage{array,mathtools}
\usepackage{cancel}

\DeclareMathOperator*{\E}{\mathbb{E}}
\let\Pr\relax
\DeclareMathOperator*{\Pr}{\mathbb{P}}
\DeclareMathOperator*{\R}{\mathbb{E}}

\topmargin 0pt
\advance \topmargin by -\headheight
\advance \topmargin by -\headsep
\textheight 8.9in
\oddsidemargin 0pt
\evensidemargin \oddsidemargin
\marginparwidth 0.5in
\textwidth 6.5in

\parindent 0in
\parskip 1.5ex

\newcommand{\homework}[2]{
	\noindent
	\begin{center}
		\framebox{
			\vbox{
				\hbox to 6.50in { {\bf NYU Computer Science Bridge to Tandon Course} \hfill Winter 2021 }
				\vspace{4mm}
				\hbox to 6.50in { {\Large \hfill Homework #1  \hfill} }
				\vspace{2mm}
				\hbox to 6.50in { {Name: #2 \hfill} }
			}
		}
	\end{center}
	\vspace*{4mm}
}

\def\unsignedbytecalc#1{%
\par\smallskip
\smallskip
\gdef\result{}%
$\left.\begin{array}{r@{\quad}|c}\udbc{#1}\end{array}\right\}\mathbf{\result_2}$\par}

\makeatletter

\def\udbc#1{%
\ifnum#1=\z@
\expandafter\@gobble
\else
\expandafter\@firstofone
\fi
{2)\!\underline{\,#1}&\edef\r{\ifodd#1 1\else 0\fi}\r\xdef\result{\r\result}\\
\expandafter\udbc\expandafter{\the\numexpr(\ifodd#1 #1-1\else#1\fi)/2\relax}%
}}

\newcommand*{\carry}[1][1]{\overset{#1}}
\newcolumntype{B}[1]{r*{#1}{@{\,}r}}
\renewcommand*{\proofname}{Solution}
\renewcommand\qedsymbol{$\blacksquare$}

\begin{document}
	
	\homework{1}{Yun-Ping Du}
	
	\section*{Question 1}
	\begin{enumerate}[A.]
        \item Convert the following numbers to their decimal representation. Show your work.
	    \begin{enumerate}[1.]
	        \item $10011011_2$
	            \begin{proof}
	                \begin{equation*}
                        \begin{split}
                            10011011_2 &= 2^7 + 2^4 + 2^3 + 2^1 + 2^0 \\
                                       &= 128 + 16 + 8 + 2 + 1 \\
                                       &= \mathbf{155}
                        \end{split}
                    \end{equation*}
	            \end{proof}
            \item $456_7$
                \begin{proof}
                    \begin{equation*}
                        \begin{split}
                            456_7 &= 4 \times 7^2 + 5 \times 7^1 + 6 \times 7^0 \\
                                  &= 196 + 35 + 6 \\
                                  &= \mathbf{237}
                        \end{split}
                    \end{equation*}
                \end{proof}
            \item $38\text{A}_{16}$
                \begin{proof}
                    \begin{equation*}
                        \begin{split}
                            38\text{A}_{16} &= 3 \times 16^2 + 8 \times 16^1 + 10 \times 16^0 \\
                                            &= 768 + 128 + 10 \\
                                            &= \mathbf{906}
                        \end{split}
                    \end{equation*}
                \end{proof}
            \item $2214_5$
            \begin{proof}
                \begin{equation*}
                    \begin{split}
                        2214_5 &= 2 \times 5^3 + 2 \times 5^2 + 1 \times 5^1 + 4 \times 5^0 \\
                               &= 250 + 50 + 5 + 4 \\
                               &= \mathbf{309}
                    \end{split}
                \end{equation*}
            \end{proof}
	    \end{enumerate}
	    \item Convert the following numbers to their binary representation:
	    \begin{enumerate}[1.]
	        \item $69_{10}$
                \begin{proof}
                    Recursively divide the quotients and collect the remainders
                    \begin{center}
        	            \unsignedbytecalc{69}
                    \end{center}
                    Thus the binary representation of $69_{10}$ is $\mathbf{1000101_2}$.
        	    \end{proof}
	        \item $485_{10}$
	            \begin{proof}
	                Recursively divide the quotients and collect the remainders
        	        \begin{center}
        	            \unsignedbytecalc{485}
        	        \end{center}
        	        Thus the binary representation of $485_{10}$ is $\mathbf{111100101_2}$.
	            \end{proof}
	        \item $6\text{D}1\text{A}_{16}$
	            \begin{proof}
	                We first obtain the binary representation of each digit
        	        \begin{equation*}
                        \begin{split}
                            6_{16} &= 0110_2 \\
                            \text{D}_{16} &= 1101_2 \\
                            1_{16} &= 0001_2 \\
                            \text{A}_{16} &= 1010_2 \\
                        \end{split}
                    \end{equation*}
                    Thus the binary representation of $6\text{D}1\text{A}_{16}$ is $\mathbf{110110100011010_2}$.
	            \end{proof}
	    \end{enumerate}
	    
	    \newpage
	    
	    \item Convert the following numbers to their hexadecimal representation:
	    \begin{enumerate}[1.]
	        \item $1101011_2$
	            \begin{proof}
	                Since we have
        	        \begin{align*}
        	            0110_2 &= 6_{16} \\
        	            1011_2 &= \text{B}_{16}
        	        \end{align*}
        	        thus the hexadecimal representation of $1101011_2$ is $\mathbf{6B_{16}}$.
	            \end{proof}
	        \item $895_{10}$
	            \begin{proof}
	                Recursively divide the quotients and collect the remainders
        	        \begin{align*}
        	            895 \div 16 &= 55 \text{ R} 15 \rightarrow \text{F}_{16} \\
        	            55 \div 16 &= 3 \text{ R} 7 \rightarrow 7_{16} \\
        	            3 \div 16 &= 0 \text{ R} 3 \rightarrow 3_{16}
        	        \end{align*}
        	        Thus the hexadecimal representation of $895_{10}$ is $\mathbf{37F_{16}}$.
	            \end{proof}
	    \end{enumerate}
	\end{enumerate}
	
	\newpage

	\section*{Question 2}
	Solve the following, do all calculation in the given base. Show your work.
	\begin{enumerate}[1.]
	    \item $7566_8 + 4515_8 =$
	        \begin{proof}
        	    \begin{equation*}
                \begin{array}{B1}
                    \carry 7 \carry 5 \carry 66_8 \\
                    {} + 4515_8 \\ \hline
                    14303_8 \\
                \end{array}
                \end{equation*}
	        \end{proof}
	    \item $10110011_2 + 1101_2 =$
	    \begin{proof}
	        \begin{equation*}
            \begin{array}{B1}
                1 \carry 0 \carry 1 \carry 1 \carry 0 \carry 0 \carry 11_2 \\
                {} + 00001101_2 \\ \hline
                11000000_2 \\
            \end{array}
            \end{equation*}
	    \end{proof}
	    \item $7\text{A}66_{16} + 45\text{C}5_{16} =$
	        \begin{proof}
    	        \begin{equation*}
                \begin{array}{B1}
                    \mathrm{\carry 7 \carry A66_{16}} \\
                    {} + \mathrm{45C5_{16}} \\ \hline
                    \mathrm{C02B_{16}} \\
                \end{array}
                \end{equation*}
	        \end{proof}
	    \item $3022_5 - 2433_5 =$
	        \begin{proof}
    	        \begin{equation*}
                \begin{array}{B1}
                    \carry[2]{\cancel{3}} \carry[4]{\cancel{0}} \carry[1]{\cancel{2}} 2_5 \\
                    {} - 2433_5 \\ \hline
                    34_5 \\
                \end{array}
                \end{equation*}
	        \end{proof}
	\end{enumerate}
	
	\newpage
	
	\section*{Question 3}
	\begin{enumerate}[A.]
	    \item Convert the following numbers to their 8-bit two’s complement representation. Show your work.
	    \begin{enumerate}[1.]
	        \item $124_{10}$
	            \begin{proof}
	                We first obtain the binary representation of $124_{10}$
        	        \begin{center}
        	            \unsignedbytecalc{124}
        	        \end{center}
        	        Thus the 8-bit two's complement representation of $124_{10}$ is $\mathbf{01111100}$.
	            \end{proof}
	        \item $-124_{10}$
	            \begin{proof}
        	        To get the 8-bit two's complement representation of a negative number, we first "flip" the bits (swapping 0s and 1s) of its positive counterpart; the value of 1 is then added to the resulting value (ignoring the overflow). We already know that $124_{10}$ is represented by
        	        \[
        	            01111100
        	        \]
        	        Flip the bits and we have
        	        \[
        	            01111100 \rightarrow 10000011
        	        \]
        	        Finally, add 1 to the representation and we have
        	        \[
        	            10000011_2 + 1_2 = 10000100_2
        	        \]
        	        Thus the 8-bit two's complement representation of $-124_{10}$ is $\mathbf{10000100}$.
	            \end{proof}
	        \item $109_{10}$
	            \begin{proof}
	                We first obtain the binary representation of $109_{10}$
        	        \begin{center}
        	            \unsignedbytecalc{109}
        	        \end{center}
        	        Thus the 8-bit two's complement representation of $109_{10}$ is $\mathbf{01101101}$.
	            \end{proof}
	            
	       \newpage

	        \item $-79_{10}$
    	        \begin{proof}
    	            We first obtain the binary representation of $79_{10}$
    	            \begin{center}
    	                \unsignedbytecalc{79}
    	            \end{center}
    	            Flip the bits and we have
    	            \[
    	                01001111 \rightarrow 10110000
    	            \]
    	            Finally, add 1 to the representation and we have
    	            \[
    	                10110000_2 + 1_2 = 10110001_2
    	            \]
    	            Thus the 8-bit two's complement representation of $-79_{10}$ is $\mathbf{10110001}$.
    	        \end{proof}
	    \end{enumerate}
	    \item Convert the following numbers (represented as 8-bit two’s complement) to their decimal representation. Show your work.
	    \begin{enumerate}[1.]
	        \item $00011110_{\text{8-bit 2’s comp}}$
	            \begin{proof}
	                Since the leftmost bit is $0$, we directly convert the binary representation to the decimal representation
	                \begin{equation*}
                        \begin{split}
                            11110_2 &= 2^4 + 2^3 + 2^2 + 2^1 \\
                                    &= 16 + 8 + 4 + 2 \\
                                    &= 30
                        \end{split}
                    \end{equation*}
                    Thus the decimal representation of $00011110_{\text{8-bit 2’s comp}}$ is $\mathbf{30}$.
	            \end{proof}
	        \item $11100110_{\text{8-bit 2’s comp}}$
	            \begin{proof}
	                Since the leftmost bit is ${1}$, we know that the number is negative. To convert a negative number into its decimal representation, we first subtract 1 from the 8-bit two's complement, flip the bits of the result, and then convert the flipped bits to its decimal representation.
	                \[
	                   11100110_2 - 1_2 = 11100101_2
	                \]
	                Flip the bits
	                \[
	                    11100101 \rightarrow 00011010
	                \]
	                Finally we have
	                \begin{equation*}
                        \begin{split}
                            11010_2 &= 2^4 + 2^3 + 2^1 \\
                                    &= 16 + 8 + 2 \\
                                    &= 26
                        \end{split}
                    \end{equation*}
                    Thus the decimal representation of $11100110_{\text{8-bit 2’s comp}}$ is $\mathbf{-26}$.
	            \end{proof}
	        \item $00101101_{\text{8-bit 2’s comp}}$
	            \begin{proof}
	                Since the leftmost bit is $0$, we directly convert the binary representation to the decimal representation
	                \begin{equation*}
                        \begin{split}
                            101101_2 &= 2^5 + 2^3 + 2^2 + 2^0 \\
                                     &= 32 + 8 + 4 + 1 \\
                                     &= 45
                        \end{split}
                    \end{equation*}
                    Thus the decimal representation of $00101101_{\text{8-bit 2’s comp}}$ is $\mathbf{45}$.
	            \end{proof}
	        \item $10011110_{\text{8-bit 2’s comp}}$
	            \begin{proof}
	                Since the leftmost bit is ${1}$, we know that the number is negative. To convert a negative number into its decimal representation, we first subtract 1 from the 8-bit two's complement, flip the bits of the result, and then convert the flipped bits to its decimal representation.
	                \[
	                   10011110_2 - 1_2 = 10011101_2
	                \]
	                Flip the bits
	                \[
	                    10011101 \rightarrow 01100010
	                \]
	                Finally we have
	                \begin{equation*}
                        \begin{split}
                            1100010_2 &= 2^6 + 2^5 + 2^1 \\
                                      &= 64 + 32 + 2 \\
                                      &= 98
                        \end{split}
                    \end{equation*}
                    Thus the decimal representation of $11100110_{\text{8-bit 2’s comp}}$ is $\mathbf{-98}$.
	            \end{proof}
	    \end{enumerate}
	\end{enumerate}
	
	\newpage
	
	\section*{Question 4}
	 Solve the following questions from the Discrete Math zyBook:
	 
	 \begin{enumerate}[1.]
	     \item  Exercise 1.2.4
	     \begin{enumerate}[(a)]
	         \setcounter{enumii}{1}
	         \item
	            \begin{proof}
	                \hfill
	                \begin{center}
                        \begin{tabular}{ |c|c|c| } 
                        \hline
                        $p$ & $q$ & $\neg (p \lor q)$ \\
                        \hline
                        T & T & F \\ 
                        T & F & F \\
                        F & T & F \\
                        F & F & T \\
                        \hline
                    \end{tabular}
                    \end{center}
	            \end{proof}
	         \item
	            \begin{proof}
	                \hfill
	                \begin{center}
                        \begin{tabular}{ |c|c|c|c| } 
                        \hline
                        $p$ & $q$ & $r$ & $r \lor (p \land \neg q)$ \\
                        \hline
                        T & T & T & T \\ 
                        T & T & F & F \\
                        T & F & T & T \\
                        T & F & F & T \\
                        F & T & T & T \\ 
                        F & T & F & F \\
                        F & F & T & T \\
                        F & F & F & F \\
                        \hline
                    \end{tabular}
                    \end{center}
	            \end{proof}
	     \end{enumerate}
	     \item  Exercise 1.3.4
	     \begin{enumerate}[(a)]
	         \setcounter{enumii}{1}
	         \item
	            \begin{proof}
	                \hfill
	                \begin{center}
                        \begin{tabular}{ |c|c|c| } 
                        \hline
                        $p$ & $q$ & $(p \rightarrow q) \rightarrow (q \rightarrow p)$ \\
                        \hline
                        T & T & T \\ 
                        T & F & T \\
                        F & T & F \\
                        F & F & T \\
                        \hline
                    \end{tabular}
                    \end{center}
	            \end{proof}
	         \setcounter{enumii}{3}
	         \item
	            \begin{proof}
	                \hfill
	                \begin{center}
	                    \begin{tabular}{ |c|c|c| }
	                    \hline
	                    $p$ & $q$ & $(p \leftrightarrow q) \oplus (p \leftrightarrow \neg q)$ \\
	                    \hline
	                    T & T & T \\
	                    T & F & T \\
	                    F & T & T \\
	                    F & F & T \\
	                    \hline
	                    \end{tabular}
	                \end{center}
	            \end{proof}
	     \end{enumerate}
	 \end{enumerate}
	 
	 \newpage
	 
	 \section*{Question 5}
	  Solve the following questions from the Discrete Math zyBook:
	  
	  \begin{enumerate}[1.]
	      \item Exercise 1.2.7
	      \begin{enumerate}[(a)]
	          \setcounter{enumii}{1}
	          \item
	            \begin{proof}
	                $(B \land D) \lor (B \land M) \lor (D \land M)$
	            \end{proof}
	          \item
	            \begin{proof}
	                $B \lor (D \land M)$
	            \end{proof}
	      \end{enumerate}
	      \item Exercise 1.3.7
	      \begin{enumerate}[(a)]
	          \setcounter{enumii}{1}
	          \item
	            \begin{proof}
	                $(s \lor v) \rightarrow p$
	            \end{proof}
	          \item
	            \begin{proof}
	                $p \rightarrow y$
	            \end{proof}
	          \item
	            \begin{proof}
	                $p \leftrightarrow (s \land y)$
	            \end{proof}
	          \item
	            \begin{proof}
	                $p \rightarrow (s \lor y)$
	            \end{proof}
	      \end{enumerate}
	      \item Exercise 1.3.9
	      \begin{enumerate}[(a)]
	          \setcounter{enumii}{2}
	          \item
	            \begin{proof}
	                $c \rightarrow p$
	            \end{proof}
	          \item
	            \begin{proof}
	                $c \rightarrow p$
	            \end{proof} 
	      \end{enumerate}
	  \end{enumerate}
	  
	  \newpage
	  
	  \section*{Question 6}
	   Solve the following questions from the Discrete Math zyBook:
	   
	   \begin{enumerate}[1.]
	       \item Exercise 1.3.6
	       \begin{enumerate}[(a)]
	           \setcounter{enumii}{1}
	           \item
	             \begin{proof}
	                 If Joe is eligible for the honors program, then he maintains a B average.
	             \end{proof}
	           \item
	             \begin{proof}
	                 If Rajiv can go on the roller coaster, then he is at least four feet tall.
	             \end{proof}
	           \item
	             \begin{proof}
	                 If Rajiv is at least four feet tall, he can go on the roller coaster.
	             \end{proof}
	       \end{enumerate}
	       \item Exercise 1.3.10 ($p$: T, $q$: False, $r$: unknown)
	       \begin{enumerate}[(a)]
	           \setcounter{enumii}{2}
	           \item $(p \lor r) \leftrightarrow (q \land r)$
	             \begin{proof}
	                 False. $(p \lor r)$ is true because $p$ is true. $(q \land r)$ is false because $q$ is false. Thus the biconditional proposition is false regardless of the truth value of $r$.
	             \end{proof}
	           \item $(p \land r) \leftrightarrow (q \land r)$
	             \begin{proof}
	                 Unknown. Since $q$ is false, $(q \land r)$ is false. However, the truth value of $(p \land r)$ depends on the truth value of $r$.
	             \end{proof}
	           \item $p \rightarrow (r \lor q)$
	             \begin{proof}
	                 Unknown. The hypothesis is true, but the truth value of the conclusion depends on the truth value of $r$.
	             \end{proof}
	           \item $(p \land q) \rightarrow r$
	             \begin{proof}
	                 True. The hypothesis is false, thus the conditional proposition is true regardless of the truth value of $r$.
	             \end{proof}
	       \end{enumerate}
	   \end{enumerate}
	   
	   \newpage
	   
	   \section*{Question 7}
	   Solve Exercise 1.4.5, sections b – d, from the Discrete Math zyBook:
	   
	   \begin{enumerate}[(a)]
	       \setcounter{enumi}{1}
	       \item 
	         \begin{itemize}
	             \item If Sally did not get the job, then she was late for interview or did not update her resume.
	             \item If Sally updated her resume and was not late for her interview, then she got the job.
	         \end{itemize}
	         \begin{proof}
	             Logically equivalent.
	             \begin{itemize}
	               \item $\neg j \rightarrow (l \lor \neg r)$
	               \item $(r \land \neg l) \rightarrow j$
	             \end{itemize}
	             \begin{center}
                 \begin{tabular}{ |c|c|c|c|c| } 
                 \hline
                 $j$ & $l$ & $r$ & $\neg j \rightarrow (l \lor \neg r)$ & $(r \land \neg l) \rightarrow j$ \\
                 \hline
                 T & T & T & T & T \\ 
                 T & T & F & T & T \\
                 T & F & T & T & T \\
                 T & F & F & T & T \\
                 F & T & T & T & T \\ 
                 F & T & F & T & T \\
                 F & F & T & F & F \\
                 F & F & F & T & T \\
                 \hline
                 \end{tabular}
                 \end{center}
	         \end{proof}
	       \item 
	         \begin{itemize}
	             \item If Sally got the job then she was not late for her interview.
	             \item If Sally did not get the job, then she was late for her interview.
	         \end{itemize}
	         \begin{proof}
	             Not logically equivalent.
	             \begin{itemize}
	               \item $j \rightarrow \neg l$
	               \item $\neg j \rightarrow l$
	             \end{itemize}
	             \begin{center}
                 \begin{tabular}{ |c|c|c|c| } 
                 \hline
                 $j$ & $l$ & $j \rightarrow \neg l$ & $\neg j \rightarrow l$ \\
                 \hline
                 T & T & F & T \\ 
                 T & F & T & T \\
                 F & T & T & T \\
                 F & F & T & F \\
                 \hline
                 \end{tabular}
                 \end{center}
	         \end{proof}
	       \newpage
	       \item 
	         \begin{itemize}
	             \item If Sally updated her resume or she was not late for her interview, then she got the job.
	             \item If Sally got the job, then she updated her resume and was not late for her interview.
	         \end{itemize}
	         \begin{proof}
	             Not logically equivalent.
	             \begin{itemize}
	               \item $(r \lor \neg l) \rightarrow j$
	               \item $j \rightarrow (r \land \neg l)$
	             \end{itemize}
	             \begin{center}
                 \begin{tabular}{ |c|c|c|c|c| } 
                 \hline
                 $j$ & $l$ & $r$ & $(r \lor \neg l) \rightarrow j$ & $j \rightarrow (r \land \neg l)$ \\
                 \hline
                 T & T & T & T & F \\ 
                 T & T & F & T & F \\
                 T & F & T & T & T \\
                 T & F & F & T & F \\
                 F & T & T & F & T \\ 
                 F & T & F & T & T \\
                 F & F & T & F & T \\
                 F & F & F & F & T \\
                 \hline
                 \end{tabular}
                 \end{center}
	         \end{proof}
	   \end{enumerate}
	   
	   \newpage
	   \section*{Question 8}
	   Solve the following questions from the Discrete Math zyBook:
	   \begin{enumerate}[1.]
	       \item  Exercise 1.5.2
	       \begin{enumerate}[(a)]
	           \setcounter{enumii}{2}
	           \item $(p \rightarrow q) \land (p \rightarrow r) \equiv p \rightarrow (q \land r)$
	             \begin{proof}
	                 \begin{align*}
	                     (p \rightarrow q) \land (p \rightarrow r) &\equiv (\neg p \lor q) \land (\neg p \lor r) \\
	                     &\equiv  \neg p \lor (q \land r) \\
	                     &\equiv p \rightarrow (q \land r)
	                 \end{align*}
	             \end{proof}
	           \setcounter{enumii}{5}
	           \item $\neg(p \lor (\neg p \land q)) \equiv \neg p \land \neg q$
	             \begin{proof}
	                 \begin{align*}
	                     \neg(p \lor (\neg p \land q)) &\equiv \neg((p \lor \neg p ) \land (p \lor q)) \\
	                     &\equiv \neg(T \land (p \lor q)) \\
	                     &\equiv \neg(p \lor q) \\
	                     &\equiv \neg p \land \neg q
	                 \end{align*}
	             \end{proof}
	           \setcounter{enumii}{8}
	           \item $(p \land q) \rightarrow r \equiv (p \land \neg r) \rightarrow \neg q$
	             \begin{proof}
	                 \begin{align*}
	                     (p \land q) \rightarrow r &\equiv \neg (p \land q) \lor r \\
	                     &\equiv (\neg p \lor \neg q) \lor r \\
	                     &\equiv \neg p \lor r \lor \neg q \\
	                     &\equiv \neg (p \land \neg r) \lor \neg q \\
	                     &\equiv (p \land \neg r) \rightarrow \neg q
	                 \end{align*}
	             \end{proof}
	       \end{enumerate}
	       \item  Exercise 1.5.3
	       \begin{enumerate}[(a)]
	           \setcounter{enumii}{2}
	           \item $\neg r \land (\neg r \rightarrow p)$
	             \begin{proof}
	                 \begin{align*}
	                     \neg r \land (\neg r \rightarrow p) &\equiv \neg r \lor (\neg (\neg r) \lor p) \\
	                     &\equiv \neg r \lor (r \lor p) \\
	                     &\equiv (\neg r \lor r) \lor p \\
	                     &\equiv \text{T} \lor p \\
	                     &\equiv \text{T}
	                 \end{align*}
	             \end{proof}
	           \item $\neg(p \rightarrow q) \rightarrow \neg q$
	             \begin{proof}
	                 \begin{align*}
	                     \neg(p \rightarrow q) \rightarrow \neg q &\equiv \neg (\neg p \lor q) \rightarrow \neg q \\
	                     &\equiv \neg\neg (\neg p \lor q) \lor \neg q \\
	                     &\equiv \neg p \lor q \lor \neg q \\
	                     &\equiv \neg p \lor \text{T} \\
	                     &\equiv \text{T}
	                 \end{align*}
	             \end{proof}
	       \end{enumerate}
	   \end{enumerate}
	   
	   \newpage
	   \section*{Question 9}
	   Solve the following questions from the Discrete Math zyBook:
	   \begin{enumerate}[1.]
	       \item  Exercise 1.6.3
	       \begin{enumerate}[(a)]
	           \setcounter{enumii}{2}
	           \item 
	             \begin{proof}
	                 $\exists x \: (x = x^2)$
	             \end{proof}
	           \item 
	             \begin{proof}
	                 $\forall x \: (x \leq x^2)$
	             \end{proof}
	       \end{enumerate}
	       \item  Exercise 1.7.4
	       \begin{enumerate}[(a)]
	           \setcounter{enumii}{1}
	           \item 
	             \begin{proof}
	                 $\forall x \: (\neg S(x) \land W(x))$
	             \end{proof}
	           \item 
	             \begin{proof}
	                 $\forall x \: (S(x) \rightarrow \neg W(x))$
	             \end{proof}
	           \item 
	             \begin{proof}
	                 $\exists x \: (S(x) \land W(x))$
	             \end{proof}
	       \end{enumerate}
	   \end{enumerate} 
	   
	   \newpage
	   \section*{Question 10}
	   Solve the following questions from the Discrete Math zyBook:
	   \begin{enumerate}[1.]
	       \item  Exercise 1.7.9
	       \begin{center}
	           \begin{tabular}{|c|c|c|c|}
	               \hline
	                   &  $P(x)$ & $Q(x)$ & $R(x)$ \\
	               \hline
	               $a$ & T & T & F \\
	               $b$ & T & F & F \\
	               $c$ & F & T & F \\
	               $d$ & T & T & F \\
	               $e$ & T & T & T \\
	               \hline
	           \end{tabular}
	       \end{center}
	       \begin{enumerate}[(a)]
	           \setcounter{enumii}{2}
	           \item $\exists x \: ((x = c) \rightarrow P(x))$
	             \begin{proof}
	                 True. Example: $a$.
	             \end{proof}
	           \item $\exists x \: (Q(x) \land R(x))$
	             \begin{proof}
	                 True. Example: $e$.
	             \end{proof}
	           \item $Q(a) \land P(d)$
	             \begin{proof}
	                 True.
	             \end{proof}
	           \item $\forall x \: ((x \neq b) \rightarrow Q(x))$
	             \begin{proof}
	                 True.
	             \end{proof}
	           \item $\forall x \: (P(x) \lor R(x))$
	             \begin{proof}
	                 False. Counterexample: $c$.
	             \end{proof}
	           \item $\forall x \: (R(x) \rightarrow P(x))$
	             \begin{proof}
	                 True.
	             \end{proof}
	           \item $\exists x \: (Q(x) \lor R(x))$
	             \begin{proof}
	                 True. Example: $a$.
	             \end{proof}
	       \end{enumerate}
	       \newpage
	       \item  Exercise 1.9.2
	       \begin{center}
	           \begin{tabular}{|c|c|c|c|}
	           \hline
                $P$ & 1 & 2 & 3 \\
               \hline
                 1  & T & F & T \\
                 2  & T & F & T \\
                 3  & T & T & F \\
               \hline
	           \end{tabular}
	           \begin{tabular}{|c|c|c|c|}
	           \hline
                $Q$ & 1 & 2 & 3 \\
               \hline
                 1  & F & F & F \\
                 2  & T & T & T \\
                 3  & T & F & F \\
               \hline
	           \end{tabular}
	           \begin{tabular}{|c|c|c|c|}
	           \hline
                $S$ & 1 & 2 & 3 \\
               \hline
                 1  & F & F & F \\
                 2  & F & F & F \\
                 3  & F & F & F \\
               \hline
	           \end{tabular}
	       \end{center}
	       \begin{enumerate}[(a)]
	           \setcounter{enumii}{1}
	           \item $\exists x \: \forall y \: Q(x, y)$
	             \begin{proof}
	                 True. Let $x = 2$, then $Q(x, 1)$, $Q(x, 2)$, and $Q(x, 3)$ are all true.
	             \end{proof}
	           \item $\exists x \: \forall y \: P(y, x)$
	             \begin{proof}
	                 True. Let $x = 1$, then $P(1, x)$, $P(2, x)$, and $P(3, x)$ are all true.
	             \end{proof}
	           \item $\exists x \: \exists y \: S(x, y)$
	             \begin{proof}
	                 False. There is no pair $(x, y)$ such that $S(x, y)$ is true.
	             \end{proof}
	           \item $\forall x \: \exists y \: Q(x, y)$
	             \begin{proof}
	                 False. Let $x = 1$, then there is no $y$ such that $Q(1, y)$ is true.
	             \end{proof}
	           \item $\forall x \: \exists y \: P(x, y)$
	             \begin{proof}
	                 True. When $x = 1$, let $y = 1$. When $x = 2$, let $y = 1$. When $3 = 1$, let $y = 1$
	             \end{proof}
	           \item $\forall x \: \forall y \: P(x, y)$
	             \begin{proof}
	                 False. Counterexample: $(x, y) = (1, 2)$.
	             \end{proof}
	           \item $\exists x \: \exists y \: Q(x, y)$
	             \begin{proof}
	                 True. Example: $(x, y) = (2, 1)$.
	             \end{proof}
	           \item $\forall x \: \forall y \: \neg S(x, y)$
	             \begin{proof}
	                 True.
	             \end{proof}
	       \end{enumerate}
	   \end{enumerate}
	   
	   \newpage
	   \section*{Question 11}
	   Solve the following questions from the Discrete Math zyBook:
	   \begin{enumerate}[1.]
	       \item  Exercise 1.10.4
	       \begin{enumerate}[(a)]
	           \setcounter{enumii}{2}
	           \item There are two numbers whose sum is equal to their product.
	             \begin{proof}
	                 $\exists x \: \exists y \: (x + y = xy)$
	             \end{proof}
	           \item The ratio of every two positive numbers is also positive.
	             \begin{proof}
	                 $\forall x \: \forall y \: ((x > 0 \land y > 0) \rightarrow x/y > 0)$
	             \end{proof}
	           \item The reciprocal of every positive number less than one is greater than one.
	             \begin{proof}
	                 $\forall x \: ((x > 0) \land (x < 1) \rightarrow 1/x > 1)$
	             \end{proof}
	           \item There is no smallest number.
	             \begin{proof}
	                 $\forall x \: \exists y \: (x > y)$
	             \end{proof}
	           \item Every number besides 0 has a multiplicative inverse.
	             \begin{proof}
	                 $\forall x \: \exists y \: ((x \neq 0) \rightarrow xy = 1)$
	             \end{proof}
	       \end{enumerate}
	       \item  Exercise 1.10.7
	       \begin{itemize}
	           \item $P(x, y)$: $x$ knows $y$'s phone number. (A person may or may not know their own phone number.)
	           \item $D(x)$: $x$ missed the deadline.
	           \item $N(x)$: $x$ is a new employee.
	       \end{itemize}
	       \begin{enumerate}[(a)]
	           \setcounter{enumii}{2}
	           \item There is at least one new employee who missed the deadline.
	             \begin{proof}
	                 $\exists x \: (N(x) \land D(x))$
	             \end{proof}
	           \item Sam knows the phone number of everyone who missed the deadline.
	             \begin{proof}
	                 $\forall x \: (D(x) \rightarrow P(\text{Sam}, x))$
	             \end{proof}
	           \item There is a new employee who knows everyone's phone number.
	             \begin{proof}
	                 $\exists x \: \forall y \: (N(x) \land P(x, y))$
	             \end{proof}
	           \item Exactly one new employee missed the deadline.
	             \begin{proof}
	                 $\exists x \: \forall y \: ((N(x) \land D(x)) \land (((y \neq x) \land N(y)) \rightarrow \neg D(y)))$
	             \end{proof}
	       \end{enumerate}
	       \newpage
	       \item  Exercise 1.10.10
	       \begin{enumerate}[(a)]
	           \setcounter{enumii}{2}
	           \item Every student has taken at least one class besides Math 101.
	             \begin{proof}
	                 $\forall x \: \exists y \: ((y \neq \text{Math 101}) \land T(x, y))$
	             \end{proof}
	           \item There is a student who has taken every math class besides Math 101.
	             \begin{proof}
	                 $\exists x \: \forall y \: ((y \neq \text{Math 101}) \rightarrow T(x, y))$
	             \end{proof}
	           \item Everyone besides Sam has taken at least two different math classes.
	             \begin{proof}
	                 $\forall x \: \exists y \: \exists z \: ((x \neq \text{Sam}) \rightarrow ((y \neq z) \land T(x, y) \land T(x, z)))$
	             \end{proof}
	           \item Sam has taken exactly two math classes.
	             \begin{proof}
	                 $\exists x \: \exists y \: \forall z \: ((x \neq y) \land T(\text{Sam},x) \land T(\text{Sam},y) \land (((z \neq x) \land (z \neq y)) \rightarrow \neg T(\text{Sam}, z)))$
	             \end{proof}
	       \end{enumerate}
	   \end{enumerate}
	   
	   \newpage
	   \section*{Question 12}
	   Solve the following questions from the Discrete Math zyBook:
	   \begin{enumerate}[1.]
	       \item  Exercise 1.8.2
	       \begin{itemize}
	           \item $P(x)$: $x$ was given the placebo
	           \item $D(x)$: $x$ was given the medication
	           \item $M(x)$: $x$ had migraines
	       \end{itemize}
	       \begin{enumerate}[(a)]
	           \setcounter{enumii}{1}
	           \item Every patient was given the medication or the placebo or both.
	             \begin{proof}
	             \hfill
	                 \begin{itemize}
	                     \item $\forall x \: (D(x) \lor P(x))$
	                     \item Negation: $\neg \forall x \: (D(x) \lor P(x))$
	                     \item Applying De Morgan's law: $\exists x \: (\neg D(x) \land \neg P(x))$
	                     \item English: There is a patient who was not given the medication and the placebo.
	                 \end{itemize}
	             \end{proof}
	           \item There is a patient who took the medication and had migraines.
	             \begin{proof}
	             \hfill
	                 \begin{itemize}
	                     \item $\exists x \: (D(x) \land M(x))$
	                     \item Negation: $\neg\exists x \: (D(x) \land M(x))$
	                     \item Applying De Morgan's law: $\forall x \: (\neg D(x) \lor \neg M(x))$
	                     \item English: Every patient either did not took the medication or did not have migraines (or both).
	                 \end{itemize}
	             \end{proof}
	           \item Every patient who took the placebo had migraines.
	             \begin{proof}
	             \hfill
	                 \begin{itemize}
	                     \item $\forall x \: (P(x) \rightarrow M(x))$
	                     \item Negation: $\neg\forall x \: (P(x) \rightarrow M(x))$
	                     \item Applying De Morgan's law: $\exists x \: (P(x) \land \neg M(x))$
	                     \item English: There is a patient who took the placebo and did not have migraines.
	                 \end{itemize}
	             \end{proof}
	           \newpage
	           \item There is a patient who had migraines and was given the placebo.
	             \begin{proof}
	             \hfill
	                 \begin{itemize}
	                     \item $\exists x \: (M(x) \land P(x))$
	                     \item Negation: $\neg\exists x \: (M(x) \land P(x))$
	                     \item Applying De Morgan's law: $\forall x \: (\neg M(x) \lor \neg P(x))$
	                     \item English: Every patient either did not had migraines or was not given the placebo (or both).
	                 \end{itemize}
	             \end{proof}
	       \end{enumerate}
	       \item  Exercise 1.9.4
	       \begin{enumerate}[(a)]
	           \setcounter{enumii}{2}
	           \item $\exists x \: \forall y \: (P(x, y) \rightarrow Q(x, y))$
	             \begin{proof}
	                 \begin{align*}
	                     \neg\exists x \: \forall y \: (P(x, y) \rightarrow Q(x, y)) &\equiv \forall x \: \exists y \: \neg(\neg P(x,y) \lor Q(x,y)) \\
	                     &\equiv \forall x \: \exists y \: (P(x,y) \land \neg Q(x,y))
	                 \end{align*}
	             \end{proof}
	           \item $\exists x \: \forall y \: (P(x, y) \leftrightarrow P(y, x))$
	             \begin{proof}
	                 \begin{align*}
	                     \neg\exists x \: \forall y \: (P(x, y) \leftrightarrow P(y, x)) &\equiv \forall x \: \exists y \: \neg((P(x,y) \rightarrow P(y,x)) \land ((P(y,x) \rightarrow P(x,y))) \\ 
	                     &\equiv \forall x \: \exists y \: \neg((\neg P(x,y) \lor P(y,x)) \land (\neg P(y,x) \lor P(x,y))) \\
	                     &\equiv \forall x \: \exists y \: (\neg(\neg P(x,y) \lor P(y,x)) \lor \neg(\neg P(y,x) \lor P(x,y))) \\
	                     &\equiv \forall x \: \exists y \: ((P(x,y) \land \neg P(y,x)) \lor (P(y,x) \land \neg P(x,y)))
	                 \end{align*}
	             \end{proof}
	           \item $\exists x \: \exists y \: P(x, y) \land \forall x \forall y \: Q(x, y)$
	             \begin{proof}
	                 \begin{align*}
	                     \neg(\exists x \: \exists y \: P(x, y) \land \forall x \forall y \: Q(x, y)) &\equiv \neg\exists x \: \exists y \: P(x, y) \lor \neg\forall x \forall y \: Q(x, y) \\
	                     &\equiv \forall x \: \forall y \: \neg P(x,y) \lor \exists x \: \exists y \: \neg Q(x,y)
	                 \end{align*}
	             \end{proof}
	       \end{enumerate}
	   \end{enumerate}

\end{document}
